\smalltitle{سوال 2}
\begin{enumerate}
  \item \phantom{text}
        
        \begin{latin}
          $P[X_6|Y_2,Y_3,Y_4,Y_5,Y_6,X_1,...,X_5]= \sum_{X_5}P[X_6|Y_2,...,Y_6,X_5]\cdots$
          \\\\
          % $P[X_1] = P[]$
          We would eliminate 6 variables so we would have a factor of size $2^6$
          \\ 
          
        \end{latin}
  \item \phantom{text}
        \begin{latin}
          If we first eliminate $X_2 \rightarrow X_5$ we would have a factor size of $2^2$
        \end{latin}
  \item \phantom{text}
        \begin{latin}
          If we want to find $P[x_5|Y_5,...,Y_{10}] \rightarrow x_9,x_{10},x_4,x_7,x_8,x_3,x_6,x_2,x_1$\\
          This sequence would lead to a factor size of 4
        \end{latin}
\end{enumerate}
