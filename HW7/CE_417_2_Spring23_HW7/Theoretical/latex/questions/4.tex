\smalltitle{سوال 4}
\noindent
\begin{enumerate}
    \item \phantom{text}
    \\
    مفهوم overfitting عملا به این معنا است که انقدر نمودار نهایی را در هنگام رگرسیون به داده‌ها نزدیک کنیم که نمودار کلی ما برای تخمین داده‌های بعدی اشتباه باشد.
    \\
    مفهوم underfitting نیز این است که آنقدر نمودار نهایی رگرسیون را ساده فرض کنیم که تخمین‌ غلطی از داده‌ها به ما بدهد.
    \\
    در شکل مشاهده شده می‌تواهن دید پیچیدگی زیاد مدل منجر به overfitting شده و هرچند ارور Train Set کم شده است اما ارور vaildation زیاد شده است که یعنی صرفا داریم نمودار دقیق‌تر برای Training Set می‌دهیم اما لزوما پیش‌بینی درستی نداریم.
    \\
    در سمت چپ نمودار یعنی با مدل خیلی ساده نیز همانطور که انتظار می‌رود underfitting داریم چرا که انگار یک نمودار خیلی ساده بدون در نظر گرفتن رفتارهای مدل داریم به آن نسبت می‌دهیم.

    \item \phantom{text}
    \\
    نمودار دوم چرا که اگر خیلی مدل پیچیده باشد دچار overfitting شده و اگر نیز خیلی ساده باشد دچار underfitting می‌شود که باید از هر دو اجتناب کنیم و یک حد خوب برای تخمین نمودار رگرسیون بدست آوریم.

\end{enumerate}
 