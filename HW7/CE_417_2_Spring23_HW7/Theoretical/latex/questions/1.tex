% \begin{latin}
%   Shoutout to Hirbod Behnam for helping me out on the questions.
% \end{latin}
\smalltitle{سوال 1}

\begin{enumerate}
    \item \phantom{text}
    \\
      در صورتی که یکی از ورودی‌های سافت مکس به بی‌نهایت میل کند،‌در این صورت به این دلیل که خروجی آن متشکل از تقسیم عبارات نمایی این ورودی‌ها است پس قطعا انقدر مخرج بزرگ شده که تمامی احتمالات به ۰ میل می‌کند بجز احتمالای که بی‌نهایت در صورت آن است که به ۱ میل ‌می‌کند.
    
    \item \phantom{text}
    \\
    در صورتی که یک پرسپترون خالی داشته باشیم به علت ذات غیر خطی xor نمی‌توانیم آنرا مدل کنیم اما اگر مثلا بتوانیم چند لایه perceptron داشته باشیم این کار را می‌توان انجام داد.
    \\
    برای and می‌توان تابع خطی زیر را درنظر گرفت :‌ 
    \\
    \begin{latin}
      \begin{center}
        $y = w_1 * x_1 + w_2 * x_2 + b \xrightarrow[]{w_1 = w_2 = 1 , b = -1}y = x_1 + x_2 - 1$
      \end{center}
    \end{latin}
    برای or می‌توان تابع زیر را در نظر گرفت
    \\
    \begin{latin}
      \begin{center}
        $y = w_1 * x_1 + w_2 * x_2 + b \xrightarrow[]{w_1 = w_2 = 2 , b = -1}y = 2x_1 + 2x_2 - 1$
      \end{center}
    \end{latin}
  \end{enumerate}





